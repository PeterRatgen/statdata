\documentclass{article}

\usepackage[danish]{babel}
\usepackage[utf8]{inputenc}
\usepackage{float}
\usepackage{fancyhdr}
\usepackage{amsmath}
\usepackage{color}
\usepackage{listings}
\usepackage{graphicx}
\usepackage{lastpage}
\usepackage{enumitem}
\usepackage[a4paper, top = 1in, bottom = 1in, left=1in,right=1in]{geometry}

\title{Statistik og dataanalyse}
\author{Peter Heilbo Ratgen - perat17@student.sdu.dk}
\date{\today}

\begin{document}
\maketitle
\section{Uge 36}
Denne uge er om kapitlet "Introduction to data".
\subsection{Introduktion til kurset}
Eksamen er en multiple choice. Undervejs er det tællende aktiviteter.

\subsection{Statistik Metoder}
Hvordan indsamler vi data, i forhold til hvad vi skal vide? Har vores data bias?
Vi har alle sammen bias på en eller anden led. Vi skal gerne lande et sted
mellem teori, viden og virkelighed. 

Generelt for man svar som man spørger. Hvis man putter urelaterede punkter ind
og laver regression, får man altid et matematisk svar, men om dette er korrekt
er ikke relateret til det matematiske. Statistik analyse baseret dig på normalt
distribueret data, uafhængighed og stor eller lille data størrelse. Ved ikke at
følge disse principper kan vi drage forkerte konklusioner fra dårlig data. Den
teoretisk model er korrekt nok, men vi kan ikke drage konklusioner fra dårlig
data.

En normalfordeling er den fine lille klokkekurve, fx højde, IQ, vægt, mv. Når
man har data nok vil det blive normalfordelt. Generelt set, skal man lave være
med at arbejde i små populationer.
Data må ikke kunne påvirke hinanden, uafhængighed er det vigtigste i statisk.
Det er hele præmissen for statetisk.

\subsection{Arbejde med data}
Vi skal have en stikprøve. Vi starter med en hypotese. Så skal vi finde en model
i den statistiske værktøjskasse. Nogengange ligger svære i at finde det rigtige
værktøj. Så estimerer vi, hvad vi før ud af den model vi har valgt. Mean har
selvfølgelig en forventning om hvad der skal komme ud. Derefter evaluerer vi
resultatet af modelleringen fx. har jeg fået det ud af det jeg forventede?

\subsection{Population til stikprøve}
Man skal have en stikprøve fra den samlede population. En stikprøve er korrekt
når den ikke er biased eller noget i den retning. Stikprøven skal være
repræsentativ for den samlede population. Den skal også være stokastisk, man
skal sikre sig at den man vælger, faktisk er tilfældig.
Så kan konklusion der drages af stikprøven, anvendes på den større population.

\subsection{Data}
Vi har forskellige typer af data. 

\begin{itemize}
  \item Kontinuert - numerisk
    \subitem en flydende overgang i data, fx hvor gammel nogen er. En person
    er et vidst antal år, måneder, dage, timer, sekunder, milisekunder, osv.
  \item Diskret  - numerisk
    \subitem Enkelte tal, fx en karakterrække
  \item Nominel - kategorisk
  \item Ordinær - kategorisk
\end{itemize}

Association er ikke det samme som kausalitet. Kausalitet kan kun drages fra
randomiserede eksperimenter. Hvis man bare kigger på tal og tænker sig til en
sammenhæng, kan man drage forkerte konklusioner. Et eksempel er Minnesota, med
bøgerne i trailerparkerne, der var blevet sat ud på grund af at man havde
fundet, det gik bedre for børn i hjem med bøger.

\subsection{Statistik og programmering}
  Vi kan bruge mange værktøjer til at lave statetisk. Vi bruger R. Python kan
  også bruges til den slags. R er lavet specifikt til formålet, det er lavet af
  statistikere.

\end{document}
